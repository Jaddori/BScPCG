The results from the user study and the implementation are presented here. In order to measure the viability of the cities the following three requirements were set Validity, Determinism, Performance and Flexibility, definitions can be found at \ref{sec:viability}. The images used during the user study are shown below:

\begin{figure}[h]
	\centering
	\begin{subfigure}{0.4\textwidth}
		\centering
		\includegraphics[width=0.9\linewidth]{"Images/bad parameters"}
		\caption*{City 1 with bad parameters.}
		\label{fig:bad-parameters}
	\end{subfigure}
	\begin{subfigure}{0.4\textwidth}
		\centering
		\includegraphics[width=0.9\linewidth]{"Images/good parameters 02"}
		\caption*{City 2 with good parameters.}
		\label{fig:good-parameters-02}
	\end{subfigure}
	%\caption{City 1 and 2.}
	\label{fig:city12}
\end{figure}

\begin{figure}[h]
	\centering
	\begin{subfigure}{0.4\textwidth}
		\centering
		\includegraphics[width=0.9\linewidth]{"Images/full random"}
		\caption*{City 3 generated at random.}
		\label{fig:full-random}
	\end{subfigure}
	\begin{subfigure}{0.4\textwidth}
		\centering
		\includegraphics[width=0.9\linewidth]{"Images/good parameters 03"}
		\caption*{City 4 with good parameters.}
		\label{fig:good-parameters-03}
	\end{subfigure}
	%\caption{City 3 and 4.}
	\label{fig:city34}
\end{figure}

\begin{figure}[h]
	\centering
	\includegraphics[width=0.36\textwidth]{"Images/random districts"}
	\caption*{City 5 with random districts.}
	\label{fig:random-districts}
\end{figure}
	
\section{User study}
	A user study with 25 participants was conducted. The participants answered questions about images of procedurally generated cities from the implementation created. The user study aimed to measure the viability the cities have in a game.
	
	% how often do you play?
	\par
	The participants of the user study is divided into three groups: \textit{avid gamers}, \textit{occasional gamers} and \textit{non-gamers}. By dividing the participants into levels of familiarity with games, a stronger case can be made for the validity of their answers. Figure \ref{fig:pie-chart-often-play} shows that most participants play at least once a week, putting them in the group of \textit{avid gamers}. Not enough participants fell into the group of \textit{occasional} or \textit{non-gamers}. Their results have been discarded.
	
	\begin{figure}[h]
		\centering
		\includegraphics[width=0.5\textwidth]{"Images/OftenPlay"}
		\caption{Pie chart of how often the participants play games.}
		\label{fig:pie-chart-often-play}
	\end{figure}
	
	% viable in games?
	The main focus of the user study is to determine if a city generated using our application would be viable in games. The user study asks the participants which (if any) of the cities they think could be used in a game. Figure \ref{fig:pie-chart-use-in-game} shows the participants answers to this question.
	
	\begin{figure}[h]
		\centering
		\includegraphics[width=0.5\textwidth]{"Images/UseInGame"}
		\caption{Bar chart of which cities the participants think could be used in a game.}
		\label{fig:pie-chart-use-in-game}
	\end{figure}
	
	% PCG better than random?
	Under Viability (\ref{sec:viability}), two requirements are outline. A generated city must differ significantly from a city generated at random. A generated city must also avoid breaking the users immersion. In an effort to determine if our generation process fulfills these requirements, the participants were asked which city looks the most random and which city looks the most repetitive. The results are shown in Figure \ref{fig:pie-chart-random-repetitive}.
	
	\begin{figure}[h]
		\begin{subfigure}{0.5\textwidth}
			\centering
			\includegraphics[width=0.95\linewidth]{"Images/Random"}
			\caption{Random cities chart.}
			\label{fig:pie-chart-random}
		\end{subfigure}
		\begin{subfigure}{0.5\textwidth}
			\centering
			\includegraphics[width=0.95\linewidth]{"Images/Repetitive"}
			\caption{Repetitive cities chart.}
			\label{fig:pie-chart-repetitive}
		\end{subfigure}
		\caption{Pie charts of random and repetitive cities.}
		\label{fig:pie-chart-random-repetitive}
	\end{figure}
	
	% least/most natural
	Figure \ref{fig:pie-chart-natural-least-natural} shows the participants answers to the question about which city looks the least and most natural. This question was intended to clarify the impact of districts on feel of the city.
	
	\begin{figure}[h]
		\begin{subfigure}{0.5\textwidth}
			\centering
			\includegraphics[width=0.95\linewidth]{"Images/LeastNatural"}
			\caption{Least natural cities chart.}
			\label{fig:pie-chart-least-natural}
		\end{subfigure}
		\begin{subfigure}{0.5\textwidth}
			\centering
			\includegraphics[width=0.95\linewidth]{"Images/Natural"}
			\caption{Natural cities chart.}
			\label{fig:pie-chart-natural}
		\end{subfigure}
		\caption{Pie charts of most and least natural cities.}
		\label{fig:pie-chart-natural-least-natural}
	\end{figure}
	
	% best roads
	\newpage
	Another fundamental part of the generation process is the road network. To determine the impact that this stage has on the city, we asked the participants which of the cities had the best road network. Figure \ref{fig:pie-chart-road-network} shows the participants answers to this question.
	
	\begin{figure}[h]
		\centering
		\includegraphics[width=0.6\textwidth]{"Images/RoadNetwork"}
		\caption{Pie chart of which cities has the best road network.}
		\label{fig:pie-chart-road-network}
	\end{figure}
	
	% ugliest/prettiest
	By asking a broad question about the look of the cities and cross referencing the result against their perceived viability in games, a correlation between look and viability can be determined. This is the intention behind asking the participants which of the cities looks the ugliest and which one looks the prettiest. Figure \ref{fig:pie-chart-ugly-pretty} shows the result of these questions.
	
	\begin{figure}[h]
		\begin{subfigure}{0.5\textwidth}
			\centering
			\includegraphics[width=0.9\linewidth]{"Images/Ugliest"}
			\caption{Ugliest cities chart.}
			\label{fig:pie-chart-ugly}
		\end{subfigure}
		\begin{subfigure}{0.5\textwidth}
			\centering
			\includegraphics[width=0.9\linewidth]{"Images/Prettiest"}
			\caption{Prettiest cities chart.}
			\label{fig:pie-chart-pretty}
		\end{subfigure}
		\caption{Pie charts of prettiest and ugliest cities.}
		\label{fig:pie-chart-ugly-pretty}
	\end{figure}


\newpage
\section{Data collection}
	The following data were collected generating cities using the implementation. The parameters controlling the city were all randomly set during each generation. The cities were generated 100 times for each size respectively. The variable measured are all the average number of the 100 samples. The generation time is measured in seconds, the size is measured in cells with a width and a height. The table is read from left to the right.
	
	\begin{center}
		\begin{tabular}{||c | c c c c||} 
			\hline
			Size & Generation time & Perlin noise calls & Buildings & Vacant \\ [0.9ex] 
			\hline\hline
			10x10 & 0.0017s & 155 & 36 & 38 \\ 
			\hline
			25x25 & 0.0056s & 745 & 213 & 157 \\
			\hline
			50x50 & 0.018s & 2997 & 838 & 664 \\
			\hline
			100x100 & 0.065s & 12083 & 3534 & 2619 \\
			\hline
			250x250 & 0.378s & 75112 & 21012 & 18588 \\
			\hline
			500x500 & 1.529s & 302188 & 87591 & 72086 \\
			\hline
			1000x1000 & 6.750s & 1216841 & 367258 & 268168 \\ [2ex] 
			\hline
		\end{tabular}
	\end{center}
	
