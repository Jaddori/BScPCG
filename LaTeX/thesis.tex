\documentclass[a4paper,oneside]{bth}

\usepackage{amsmath}
\usepackage{mathenv}
\usepackage{amssymb}
\usepackage{amsthm}
\usepackage{textcomp}
\usepackage{longtable}
\usepackage{multirow}
\usepackage{pifont}
\usepackage{changepage}
\usepackage{listings}
\usepackage{url}
\usepackage{xspace}
\usepackage{xtab}
\usepackage[utf8]{inputenc}
\usepackage[T1]{fontenc}
\usepackage{graphicx}
\usepackage{subcaption}
\DeclareGraphicsExtensions{.pdf, .png, .jpg}

\newtheorem{lem}{\textsc{Lemma}}[chapter]
\newtheorem{thm}{\textsc{Theorem}}[chapter]
\newtheorem{prop}{\textsc{Proposition}}[chapter]
\newtheorem{post}{Postulate}[chapter]
\newtheorem{corr}{\textsc{Corollary}}[chapter]
\newtheorem{defs}{\textsc{Definition}}[chapter]
\newtheorem{cons}{\textsc{Constraint}}[chapter]
\newtheorem{ex}{\textbf{Example}}[chapter]
\newtheorem{qu}{\textbf{Question}}[chapter]

\newcommand\tab[1][1cm]{\hspace*{#1}} % Add \tab :D



\begin{document}

\pagestyle{plain}
\pagenumbering{roman}


% FRONT MATTER
%-------------
% Please change the data below appropriately.
% The data will be used to generate text in various places on the
% thesis front and inner pages.
%-------------
% The month year when your final report was submitted.
\newcommand{\thesisMonth}{June}
\newcommand{\thesisYear}{2017}

% The degree name you are submitting your thesis for.
% This must be one of the following:
%    Bachelor of Science in Computer Science
%    Bachelor of Science in Digital Game Development
%    Master of Science in Computer Science
%    Master of Science in Software Engineering
%    Master of Science in Telecommunication Systems
\newcommand{\thesisDegree}{Bachelor of Science in Computer Science}

% Course time in weeks. For a 15 credits course this should be 10 and
% for a 30 credits course, this should be 20 weeks.
% Note that the week figure is the same whether you work alone or in a pair.
\newcommand{\thesisWeeks}{10}


% THESIS FRONT PAGE (here you insert title, subtitle and author names)
{\pagestyle{empty}
\changepage{5cm}{1cm}{-0.5cm}{-0.5cm}{}{-2cm}{}{}{}
\noindent
\begin{tabular}{@{}p{0.75\textwidth} p{0.25\textwidth}}
\thesisDegree & \hfill\multirow{3}{*}{\bthcsnotextlogo{3cm}} \\
\thesisMonth \ \thesisYear & \\
\end{tabular}

\begin{center}

\vspace {7.5cm}

% TITLE
{\Huge\textbf{Procedural city generation\\*[0.25cm]using Perlin noise}}

\vspace {0.5cm}

% SUBTITLE (if you have one)
% You should exclude this line, if you don't have a subtitle.
%{\Large\textbf{Thesis Subtitle}}

\vspace {3cm}

% AUTHOR NAMES
% Use the first version for single authors.
% The second version is for two authors on separate rows
%{\Large\textbf{Firstname Lastname}} % first version
{\Large\textbf{Niclas Olsson \\*[0.25cm] Elias Frank}} %second version

\end{center}

\vspace*{\fill}

\noindent%
{\small 
Faculty of Computing \\
Blekinge Institute of Technology \\
SE--371 79 Karlskrona, Sweden
}

\clearpage
} % Back to \pagestyle{plain}


% THESIS INNER PAGE (contact information to authors and supervisors go gere)
{\pagestyle{empty}
\changepage{5cm}{1cm}{-0.5cm}{-0.5cm}{}{-2cm}{}{}{}
\noindent
% Do not change the text below.
{\small
This thesis is submitted to the Faculty of Computing at Blekinge Institute
of Technology in partial fulfillment of the requirements for the degree of
\thesisDegree. The thesis is equivalent to \thesisWeeks\ 
weeks of full time studies.
}

\vspace{15cm}

\noindent
\textbf{Contact Information:} \\
Author(s): \\
Niclas Olsson \\
Elias Frank \\
E-mail:\\
niod12@student.bth.se \\
elfa14@student.bth.se

\vspace{2cm}

\noindent
University advisor: \\
Hans Tap, PhD \\
Department of Creative Technologies


\vspace*{\fill}

\noindent
\begin{tabular}{@{}p{0.5\textwidth} l c l}
Faculty of Computing             & Internet & : & www.bth.se \\
Blekinge Institute of Technology & Phone    & : & +46 455 38 50 00 \\
SE--371 79 Karlskrona, Sweden    & Fax      & : & +46 455 38 50 57 \\
\end{tabular}
\clearpage
} % Back to \pagestyle{plain}

\setcounter{page}{1}

% ABSTRACT PAGE (in \pagestyle{plain})
\abstract
\begin{changemargin}{+.5cm}{+.5cm}
\noindent
\textbf{Context}. Procedural content generation is to algorithmically generate content. This has been used in games and is an important tool to create games with large amounts of content using fewer resources. This may allow small developers to create big worlds, which makes the investigation into this area interesting.\newline
\textbf{Objectives}. The Procedural generation of cities using Perlin noise is explored. The goal is to find out if a procedurally generated city using Perlin noise is viable to use in games. \newline
\textbf{Methods}. An implementation generating cities using Perlin noise has been created and a user study along with data collection tests the cities' viability in games.\newline
\textbf{Results}. The implementation succeeds with all the technical requirements such as performance and determinism. The user study shows that the cities created are perceived as viable in games. \newline
\textbf{Conclusions}. The cities generated with the implementation seems to be viable in games. The results show that the generated content are percieved as more viable than random generated cities. Furthermore the generation speed is fast enough to be used in an online setting.

\vspace {1cm}
% You can list 3-4 keywords, maximum 2 of these from the title;
% starts 1 line below the abstract.
\noindent
\textbf{Keywords:} Procedural city generation, Perlin noise, Performance, Game content

\end{changemargin}


% TABLE OF CONTENTS PAGES (generated by LaTeX using the command(s) below)
% You should uncomment the commands you need.
\tableofcontents
%\listoffigures             % in case you have them
%\listoftables              % in case you have them
%\listofalgorithms          % in case you have them
%\include{acknowledgments}  % OPTIONAL

\cleardoublepage
\pagestyle{headings}
\pagenumbering{arabic}


% THE ACTUAL THESIS STARTS HERE
\chapter{Introduction}
Exploring a huge open world environment is a desirable feature in games. But creating a big open city such as in the Grand Theft Auto series and Batman: Arkham City involves years of work for a lot of people. Making big open cities in games is simply not feasible for smaller game companies. These games all have great success with their big open worlds making a feasible generated city an attractive technique for smaller companies to be able to use big open worlds of their own.
\par
Content creation requires time and resources; these are both scarce, which is why game companies need technology to help with this process. To create large amounts of content without a big workforce algorithmically based solutions exist, Procedural Content Generation (PCG). PCG was in the past used to minimize the disk space required for games. .kkrieger is an  example of this. It has since evolved into a method to minimize workforce required for content. No man\'s sky is an example of a game using PCG to minimize workforce while maximizing content.
\par
Procedural content generation is a large subject covering many different techniques. Games such as No Man's Sky\cite{NoMansSky} are using PCG to generate an entire universe of 18 quintillion\cite{NoMansSkyplanets} different planets complete with plants, animals, animations, biomes and sounds. While other games such as Borderlands\cite{Borderlands} uses PCG to generate over 17,750,000 different guns\cite{BorderlandsWeapons}. Civilization IV\cite{CivilizationCompany}, Minecraft\cite{Minecraft} and Spelunky\cite{Spelunky} are examples of games generating different game environments both in 2D and 3D using PCG. Minecrafts world being eight times bigger than the surface of the earth\cite{MinecraftSize}. With enormous amounts of content like these examples it is understandable why handcrafting this would not be feasible and PCG is an attractive creative method. A lot of work has been done in the field of PCG and it has been used within games for a long time. One of the earliest examples of a game using PCG is Elite\cite{Elite} from 1984. But no singular PCG solution has yet to be found, this may be attributed to the vastly different areas of use.
\par
In this thesis, the possibility of procedurally generating a city with Perlin noise that would be viable within a game is investigated. An implementation attempting to create such a city will be made and evaluated through a user study along with data collection from the implementation.
	
	\section{Background}
	Perlin Noise is the PCG technique chosen to use for every single stage of the city generation. This technique was mainly chosen because of how fast it is to implement as implementation time is very important in game development.
	There are many ways to procedurally generate a city, but most of them are very complicated which often means long implementation time. From a developer's perspective there is not much use for PCG if the implementation time equals the time it would take to handcraft every single element.
	
	\section{Procedural content generation}
	Procedural content generation in games is defined as "\textit{[...] the algorithmic creation of game content with limited or direct user input}"\cite{WhatIsPCG}. This is the definition used in this thesis. But PCG has been used for many purposes by many different people. There is no clear definition of exactly what PCG is that is universally agreed upon.\cite{WhatIsPCG}
	
	PCG is also used in areas outside the gaming industry. SpeedTree \cite{SpeedTree} is using PCG methods to generate trees that are used in movies such as Avatar \cite{SpeedTreeMovies}.
	This thesis will focus on the use of PCG in games with the previous mentioned definition.
	
	\section{City generation}
	A city has many different definitions in the real world. But a city viable in games is unlike real cities. They are not a large human settlement with governments and sanitary systems. The definition of a city explored in this thesis is "\textit{A large area occupied with different kinds of houses connected through a road system.}"
	Unlike a real city, a city viable in games do not need logic, electricity connections nor gas stations. It simply needs to be an engaging place to experience a game in. With this great simplification, generation of such a city becomes far less complex.
	To generate a city viable to use in games, three different generation stages have been recognized: Districts, roads (creating blocks) and buildings. These generation stages are used in the implementation.
	
	\section{Problem statement}
	How fast the content is generated and the quality of the generated content are the main problems this thesis investigates.
	To create a city purely using Perlin noise and test if it is viable in games is explored.
	
	\section{Objectives}
	Generating a city through the PCG technique Perlin noise and testing the viability in games is the goal. The time to generate a city and the cities viability within games is the corner stones explored in this thesis. The creation of an implementation is necessary as to the best knowledge of the authors there exists no such solution available.
	The following things need to be completed for the objective to be achieved.
	
	\begin{itemize}
		\item \textbf{Perlin noise:} Everything should be generated with Perlin noise.
		With only a single PCG technique to implement, the implementation should be fast to create.
		
		\item \textbf{Implementation:} An implementation that generate viable cities. This implementation will have a user interface that allows the user to rapidly generate many different cities.
		
		\item \textbf{Data collection:} The implementation must calculate and save correct data that can be analyzed. Loading times and how many times Perlin noise is used for the generation are two types of data that are relevant.
		
		\item \textbf{User study:} After the implementation is done a user study will be conducted. This user study must have atleast 20 participants answering questions from a questionnaire.
		
		\item \textbf{Conclusion: } With the data from the user study and the implementation the conclusion that Perlin noise can generate cities viable in games can be done.
	\end{itemize} 
	
	\section{Research question}
	Can Perlin noise be used to procedurally generate a city viable in games?


\chapter{Related work}
\section{Related Work}
Procedural city generation has been done many times with different angles. Greuter et al\cite{PseudoInfiniteCities} created a method to procedurally generate "\textit{pseudo infinite cities}". This is done by placing the buildings along a grid. The position of each cell in the grid is hashed into a unique number. With this number the building is procedurally generated to be unique from all the other buildings. Because the position of the buildings decides how they look you can walk for a pseudo infinite time through this city, without ever encountering a building that has been seen before. To be able to render such a city, culling of geometry as well as controlling caching (loading and de-loading of assets) is important. (pseudo infinite cities)

\par 
M\"uller proposes a method to procedurally generate a city using extended L-systems for the roads, rule based subdivision for the lots and CGA shape grammar for the geometry[reef]. This is done in stages; first the roads are generated. The roads naturally generate blocks of land. These blocks are by rule based subdivision divided into lots which fits a house generated by shape grammar as the last stage. This was done in 97.000 lines of code over 6 years of working time. \cite{ProceduralModeling}\cite{ProceduralModeling6}

\par
M\"uller, Weber, Wonka and Gross procedurally generate a 4D city, with the fourth dimension being time. This is also done through L-systems and shape grammars. The growth of the city can be seen through time. Trying to achieve a realistic growth of a city was important here as the results are compared to real life data of city growth. \cite{4DCities} There exists even more work where cities are simulated through time. \cite{AutonomousTimeVarying}


\par
Real time generation and rendering of infinite cities using the GPU has also been done\cite{InfiniteCities}. This city is created with shape grammars. With this technique thousands of building can be rendered at 100 frames per second. Even when the camera is moving fast through the city. This is achieved with advanced culling techniques, frame-to-frame coherence, buffer management and moving important work from the CPU to the GPU.

\par
Both L-systems and CGA shape grammars for city generation have updated versions that can be used for future work. \cite{InteractiveProceduralStreet}\cite{AdvancedProceduralModel}

\par
Ken Perlins Perlin noise was created to be used as textures for modeling, since its creation it has been updated by Perlin himself and has been widely used for other means than textures such as traffic generation. \cite{PerlinNoise}\cite{TrafficGenerator}

\chapter{Theoretical framework}
\section{Theoretical Framework}
	\subsection{Noise}
		\subsubsection{Lattice noises}
		Lattice noises are simple and efficient ways of generating noise\cite{TexturingModeling} . The idea behind them is to divide a coordinate system into sections spanning between every integer coordinate. These sections are called the integer lattices. By first generating pseudorandom numbers at every lattice and then interpolating between them, noise is generated.
		
		\subsubsection{Value noise}
		Value noise is a type of lattice noise. It uses a value between minus one and one at every lattice point and interpolates between them. The key difference between value noises is what type of interpolation is used. Methods such as linear and cubic interpolation have been used\cite{TexturingModeling}.
		
		\subsubsection{Gradient noise}
		Gradient noise differs from value noise by generating gradient vectors at each lattice point, instead of raw values. To calculate the noise value of a point, three steps are taken. First, a vector is calculated from each lattice point to the point.  Then, the dot product between this vector and the gradient vector at the corresponding lattice point is calculated. Finally, the noise value is calculated by interpolating between the dot product results.
		
		\subsubsection{Perlin noise}
		Perlin noise is a type of gradient noise. It was developed by Ken Perlin in 1983 and improved in 2002\cite{PerlinNoise}. Perlin noise works in one, two and three dimensions, but for the purpose of this explanation two-dimensional Perlin noise will be used. The algorithm expects two arguments, an x and a y coordinate. These values are normalized so as to reside between two integer lattices, creating a point. A vector from each lattice point to this point is calculated. One of the gradient vectors is chosen at random based on the normalized coordinates. The dot product between this gradient vector and the vector to the point is calculated at each lattice point. The final value is calculated by interpolating between these dot products. Perlin noise interpolates between values using a proprietary ease curve called fade. It has the form \begin{math}6t^5-15t^4-10t^3\end{math}.
		\begin{figure}
			\centering
			\includegraphics[width=0.5\linewidth]{"images/fade"}
			\caption{Perlin's easy curve called fade}
			\label{fig:fade}
		\end{figure}
	
	\subsection{Online vs offline}
	There are two ways to use PCG algorithms, either offline or online. Online is when the content is generated while the game is being played or in a short loading screen just before the player can start playing. This allows content that is adapted to the individual player and semi-infinite content.(Procedural content generation in games (2016), Chaker, Noor, E-book) In the game Left 4 Dead(ref) PCG is used by analyzing the players behavior and altering the experience, this could be considered a mixture of artificial intelligence and online PCG.
	\par Offline generation is when the content is being generated before the player starts playing. This could be done by the developers before releasing the game. Offline generation is useful when generating complicated content that is too slow to generate online.
	These are the definitions used in this thesis.
	
	
	\subsection{Viability}
	When developing the implementation, there had to be requirements that will ensure that the result is viable in games.
	The following requirements have been set to make sure the result will be viable.
	
	\begin{itemize}
		\item \textbf{Validity:} The city does not have to be realistic but the city must be somewhat credible for the player to enjoy the environment. The player should not lose immersion because of broken geometry within the city.
		
		\item \textbf{Determenism:} The generation should be determenistic. This means that the exact same city can be generated again with the correct seed. Nothing should be purely random. This is important so that the users have control over the output of the implementation.
		
		\item \textbf{Performance:} The city should be generated within a reasonable time frame to be viable in an on-line setting. A game should never have loading screens for any long period of time. This implementation have a time limit of 15 seconds as Paradox Interactive AB a game company set this as a time limit in Melins and Bengtssons colaboration about procedural generation with the company. (Constrained procedural floor
		plan generation for gameenvironments) Even if this limit is excedded the implementation may still be viable in a offline setting.
		
		\item \textbf{Flexibility: } The implementation should be able to generate many different viable cities with different looks.
	\end{itemize} 

	PCG can be either \textit{feasible} or \textit{infeasible}, for the generated content to be considered feasible it must fulfill all the constraints. (Constrained Novelty Search: A Study on Game Content Generation) Constraints are game specific but the criteria is that with all constraints fulfilled the game should be playable. For a city an example of a constraint would be that there must be a way to travel everywhere in the city, i.e. no part of the city would be blocked by houses or have no roads connected to it.
	
	\subsection{Evaluation}
	To know if the implementation does generate cities viable in games the results must be evaluated.

\chapter{Method}
\section{Method}

	\subsection{City generation method}
	Most related work focus on one of two things: procedurally generating unique building meshes(TODO: ref) or procedurally generating large cities consisting of similar looking buildings(TODO: ref). This work focuses instead on the procedurally generated placement of authored meshes in varied districts. The concept of districts breaks up the repetitive and self-similar nature of city by introducing distinct changes in appearance between them. During research three stages of generation were recognized: Districts, Blocks and Buildings. Each of these steps use the result of the previous one as starting point, in a top-down, hierarchical manner. Going through these stages is intended to generate a city that feels and looks natural.

		\subsubsection{Districts}
		The generation process starts with the generation of Districts. A District is an abstract representation of an area of the city. It controls what type of buildings it contains, the minimum and maximum height of these buildings and how densely populated the area is with these buildings. With just these few variables, you can create anything from a sparse industrial district to a dense inner city. The districts have a semi-random deterministic spread across the city. There is no constraints on the distribution of the districts. This means that one seed may produce a city where 90\% is occupied by one district. There are no constraints to prevent this because this would hinder the variability of the cities generated. The districts typically have a lot of blocks where no other districts are competing. But at the borders between districts there is a lot more variation, this is refered to as \textit{border competition}. Here a single block may contain buildings from all three districts. This was done to avoid hard borders between the districts. The hard borders were concidered to look strange and may compromise the cities viability in games. The districts were deemed important as they provide variation throughout the city. (TODO: More stuffs here)
		
		\subsubsection{Blocks}
		The next stage in the process is the generation of Blocks. In this work, a Block is defined as an area enclosed by four roads. Looking at aerial photographs of large cities(TODO: ref) a certain pattern sometimes appears in the way roads are laid out. There is usually a few major roads stretched throughout the city and smaller, perpendicular roads connecting them together. We decided to mimic this style(TODO: Don't use we). Since a block is another abstract representation of an area, it is actually the roads that make them up. The first step of generating blocks is to generate the main roads running through the city. After the main roads have been generated, smaller roads are generated between, connecting them together. After all roads have been generated, the city has affectively been divided into a non-uniform grid. Each cell in this grid is a block. (TODO: This stage does not depend on the previous)
		
		\subsubsection{Buildings}
		The final stage is the generation of Buildings. Each building is made up of three meshes: the bottom, middle and top meshes. By dividing a building into sections, it is possible to assemble different pieces and generate new buildings. This is a powerful way to use combinatorics to generate more content. For example, authoring two buildings would result in two unique buildings. Whereas authoring two bottom, middle and top parts would result in eight unique buildings. Upon generation of a buildings, the attributes of its district is taken into consideration. What sections to combine, the minimum and maximum height as well as the density of buildings, is all dictated by the district.

	\subsection{Implementation}
	To explore the procedural generation of a city, a desktop application was created. The application provides the end user with an interface that allows for the configuration of variables along with some statistics. 
	
		\subsubsection{Interface}
		The core functionality of our application is in the form of a proprietary static library called Elicras. It was decided early on that there should be a separation between the graphical user interface and the functionality of the application. It is therefore possible to move Elicras into another front facing interface without any significant changes to the underlying code. Elicras is divided into three areas: Rendering, Assets and PCG. Rendering, as the name implies, handles all the OpenGL function calls. Assets handles all the loading and using of models and textures. PCG handles all the generation of procedural content. The front facing system interacts with these subsystems through Elicras.
		
		\par
		The graphical interface is implemented using Qt(TODO: ref). Qt is an abstraction layer between its supported platforms and their user interface. Using Qt it is possible to create an application once and deploy it all of its supported platforms, including Windows, OSX and Linux. It abstracts away a lot of low level(TODO: ref win32 programming) platform code and provides an easy to use, WYSIWYG(TODO: ref dictionary) graphical interface where you can create the graphical interface for you application. (TODO: Image of Qt editor).
		
		\par
		The interface is divided into three sections that are laid out horizontally. The left section is for statistics such as number of buildings in each of the districts, number of roads and the time it took to generate the city. The middle section displays the generated city. The right section of the interface contains controls for altering how the city is generated. There are three parameters per district: minimum height, maximum height and density of buildings. There is also a global parameter called the seed. The seed can alter the look of the city without altering the per-district parameters. When the end user is satisfied with the parameters, they press the 'Generate' button and a city is generated and displayed on the screen.(TODO: Image of the user interface)
		
		\subsubsection{Districts}
		The district PCG generates a 2D grid with cells where each cell contains a district number. Each cell is in the later stages filled with grass, roads or buildings. Each cell within the grid represents an area of land in the city. These areas are square shaped with a length of 10 meters which equates to a total area of 100 square meters for each cell.
		
		\par
		Each district has a starting position; this will be referred to as the districts core. The core position of each district is semi-randomly distributed throughout the city with modified Perlin noise. From the core position each district takes control over all the cells they are closest to. The distance is calculated with Euclidian distance squared. The squared distance is used because this is faster for the computer to calculate than the real distance. The equation where \textbf{A} and \textbf{B} are 2D positions:
		\begin{equation}
			EuclideanSquaredDistance = (B1 - A1)^2 + (B2 - A2)^2
		\end{equation}
		
		\par
		After the districts, have expanded from their core position the borders are clear to see. There are often straight lines where on one side of the line there are skyscrapers and on the other there are industry. To negate this effect somewhat \textit{border competition} was implemented. The algorithm finds every cell where two different districts meet (i.e. One district in the cell and a new district in the cell to the right). 
		
		\par
		Explain Perlin noise in this context
		
		\subsubsection{Blocks}
		After generating districts, the roads are generated (effectively creating blocks). The first step is to generate the main roads running through the city. Main roads always run vertically through the city, no matter what parameters are used. To figure out where the main roads should be, Perlin noise is used. By going through the cells of the map horizontally and checking if the Perlin noise for that cell is above a certain threshold, Elicras determines if that cell should be a main road or not. Since the main roads run through the whole city, the algorithm only goes through the first row of cells. If the algorithm determines that a cell should contain a main road, it sets the value of that cell to indicate that it now contains a main road. (TODO: Image console window after main roads). After generating the main roads, smaller roads are generated. This algorithm works on two levels. The first level calculates the width of the road by starting at its current position in the map and going right until it finds a main road. The next level works much the same way as the algorithm for generating the main roads, but it goes vertically instead of horizontally.(TODO: image) When the second level reaches the bottom of the map, it goes back to the first level. Road generation is complete when the end of the map is reached.(TODO: image)
		
		\subsubsection{Buildings}
		The final stage of the generation process it the generation of buildings. At this point the map contains values that indicate whether a cell is a road or a vacant slot belonging to a district. This stage goes through the whole map, skipping over any roads. For each cell that the algorithm visits, three of the districts parameters are considered: minimum height, maximum height and density. The algorithm retrieves a Perlin noise value for the cell and compares it against the density threshold to determine whether to place a building or leave the cell vacant. If a building is placed, the algorithm uses Perlin noise to choose a bottom, middle and top sections. It chooses these sections from the list of available sections that belong to the district. Since the result of Perlin noise is a value between zero and one, it is easy to multiply this result with the number of available sections in order to choose one. The final step of the algorithm determines the height of the building by multiplying the difference between the maximum height and the minimum height with the result from the Perlin noise and then adding the minimum height. (TODO: Math formula?) The results of the building generation is not stored in the map, but in a separate list of building sections that can later be rendered.(TODO: Images)

	\subsection{Data collection}
	Aside from the core functionality of the application, an effort was made to record any significant data about the generation process. The observer pattern was used at the core of this effort. In Elicras there is a DataManager and many DataHolders. Any class whose data is worth collecting, inherits from the DataHolder base class. This results in a centralized way of collecting data that is easy to use and to setup. When Elicras requests the data from the DataManager, it notifies the registered DataHolders to supply their data. The application gathers the following information about the generation process:
		\begin{itemize}
			\item Number of Perlin noise calls
			\item Number of main roads
			\item Number of small roads
			\item Number of buildings in district 1-3
			\item Total number of buildings
			\item Number of grass tiles in district 1-3
			\item Total number of grass tiles
			\item Generation time
		\end{itemize}

\chapter{Results}
The results from the user study and the implementation will be presented here. In order to measure the viability of the cities the following three requirements were set Validity, Determinism, Performance and Flexibility, definitions can be found at \ref{sec:viability}.
	
\section{User study}
A user study with 25 participants was conducted. The participants answered questions about images of procedurally generated cities from the implementation created. The user study aimed to measure the viability the cities have in a game.
	
%The user study had 25 participants. The figure \ref{fig:pie-chart-often-play} shows that most participants play games more than once a week, therefore they can be expected to have knowledge about game environments.

% how often do you play?
\par
The participants of the user study is divided into three groups: \textit{gamers}, \textit{casual gamers} and \textit{non-gamers}. By dividing the participants into levels of familiarity with games, a stronger case can be made for the validity of their answers. Figure \ref{fig:pie-chart-often-play} shows that most participants play at least once a week, putting them in the group of \textit{gamers}. Not enough participants fell into the group of casual or non-gamers. Their results have been discarded.

\begin{figure}[h]
	\centering
	\includegraphics[width=0.65\textwidth]{"Images/OftenPlay"}
	\caption{Pie chart of how often the participants play games.}
	\label{fig:pie-chart-often-play}
\end{figure}

% viable in games?
The main focus of the user study is to determine if a city generated using our application would be viable in games. The user study asks the participants which (if any) of the cities they think could be used in a game. Figure \ref{fig:pie-chart-use-in-game} shows the participants answers to this question.

\begin{figure}[h]
	\centering
	\includegraphics[width=0.7\textwidth]{"Images/UseInGame"}
	\caption{Pie chart of which cities the participants think could be used in a game.}
	\label{fig:pie-chart-use-in-game}
\end{figure}

% PCG better than random?
Under Viability (\ref{sec:viability}), two requirements are outline. A generated city must differ significantly from a city generated at random. A generated city must also avoid breaking the users immersion. In an effort to determine if our generation process fulfills these requirements, the participants were asked which city looks the most random and which city looks the most repetitive. The results are shown in Figure \ref{fig:pie-chart-random-repetitive}.

\begin{figure}[h]
	\begin{subfigure}{0.5\textwidth}
		\centering
		\includegraphics[width=0.95\linewidth]{"Images/Random"}
		\caption{Random cities chart.}
		\label{fig:pie-chart-random}
	\end{subfigure}
	\begin{subfigure}{0.5\textwidth}
		\centering
		\includegraphics[width=0.95\linewidth]{"Images/Repetitive"}
		\caption{Repetitive cities chart.}
		\label{fig:pie-chart-repetitive}
	\end{subfigure}
	\caption{Pie charts of random and repetitive cities.}
	\label{fig:pie-chart-random-repetitive}
\end{figure}

% least/most natural
Figure \ref{fig:pie-chart-natural-least-natural} shows the participants answers to the question about which city looks the least and most natural. This question was intended to clarify the impact of districts on feel of the city.

\begin{figure}[h]
	\begin{subfigure}{0.5\textwidth}
		\centering
		\includegraphics[width=0.95\linewidth]{"Images/LeastNatural"}
		\caption{Least natural cities chart.}
		\label{fig:pie-chart-least-natural}
	\end{subfigure}
	\begin{subfigure}{0.5\textwidth}
		\centering
		\includegraphics[width=0.95\linewidth]{"Images/Natural"}
		\caption{Natural cities chart.}
		\label{fig:pie-chart-natural}
	\end{subfigure}
	\caption{Pie charts of most and least natural cities.}
	\label{fig:pie-chart-natural-least-natural}
\end{figure}

% best roads
Another fundamental part of the generation process is the road network. To determine the impact that this stage has on the city, we asked the participants which of the cities had the best road network. Figure \ref{fig:pie-chart-road-network} shows the participants answers to this question.

\begin{figure}[h]
	\centering
	\includegraphics[width=0.6\textwidth]{"Images/RoadNetwork"}
	\caption{Pie chart of which cities has the best road network.}
	\label{fig:pie-chart-road-network}
\end{figure}

% ugliest/prettiest
\newpage
By asking a broad question about the look of the cities and cross referencing the result against their perceived viability in games, a correlation between look and viability can be determined. This is the intention behind asking the participants which of the cities looks the ugliest and which one looks the prettiest. Figure \ref{fig:pie-chart-ugly-pretty} shows the result of these questions.

\begin{figure}[h]
	\begin{subfigure}{0.5\textwidth}
		\centering
		\includegraphics[width=0.9\linewidth]{"Images/Ugliest"}
		\caption{Ugliest cities chart.}
		\label{fig:pie-chart-ugly}
	\end{subfigure}
	\begin{subfigure}{0.5\textwidth}
		\centering
		\includegraphics[width=0.9\linewidth]{"Images/Prettiest"}
		\caption{Prettiest cities chart.}
		\label{fig:pie-chart-pretty}
	\end{subfigure}
	\caption{Pie charts of prettiest and ugliest cities.}
	\label{fig:pie-chart-ugly-pretty}
\end{figure}

\chapter{Discussion}
\section{Discussion}
\subsection{Data and user study analysis}
\subsection{Implementation analysis}
\subsection{City analysis}

\chapter{Conclusion and future work}
\section{Conclusion}
This is our conclusion

\bibliography{citations}
\bibliographystyle{plain}
\end{document}
