\documentclass[a4paper,oneside]{bth}

\usepackage{amsmath}
\usepackage{mathenv}
\usepackage{amssymb}
\usepackage{amsthm}
\usepackage{textcomp}
\usepackage{longtable}
\usepackage{multirow}
\usepackage{pifont}
\usepackage{changepage}
\usepackage{listings}
\usepackage{url}
\usepackage{xspace}
\usepackage{xtab}
\usepackage[utf8]{inputenc}
\usepackage[T1]{fontenc}
\usepackage{graphicx}
\usepackage{subcaption}
\DeclareGraphicsExtensions{.pdf, .png, .jpg}

\newtheorem{lem}{\textsc{Lemma}}[chapter]
\newtheorem{thm}{\textsc{Theorem}}[chapter]
\newtheorem{prop}{\textsc{Proposition}}[chapter]
\newtheorem{post}{Postulate}[chapter]
\newtheorem{corr}{\textsc{Corollary}}[chapter]
\newtheorem{defs}{\textsc{Definition}}[chapter]
\newtheorem{cons}{\textsc{Constraint}}[chapter]
\newtheorem{ex}{\textbf{Example}}[chapter]
\newtheorem{qu}{\textbf{Question}}[chapter]

\newcommand\tab[1][1cm]{\hspace*{#1}} % Add \tab :D



\begin{document}

\pagestyle{plain}
\pagenumbering{roman}


% FRONT MATTER
%-------------
% Please change the data below appropriately.
% The data will be used to generate text in various places on the
% thesis front and inner pages.
%-------------
% The month year when your final report was submitted.
\newcommand{\thesisMonth}{June}
\newcommand{\thesisYear}{2017}

% The degree name you are submitting your thesis for.
% This must be one of the following:
%    Bachelor of Science in Computer Science
%    Bachelor of Science in Digital Game Development
%    Master of Science in Computer Science
%    Master of Science in Software Engineering
%    Master of Science in Telecommunication Systems
\newcommand{\thesisDegree}{Bachelor of Science in Computer Science}

% Course time in weeks. For a 15 credits course this should be 10 and
% for a 30 credits course, this should be 20 weeks.
% Note that the week figure is the same whether you work alone or in a pair.
\newcommand{\thesisWeeks}{10}


% THESIS FRONT PAGE (here you insert title, subtitle and author names)
{\pagestyle{empty}
\changepage{5cm}{1cm}{-0.5cm}{-0.5cm}{}{-2cm}{}{}{}
\noindent
\begin{tabular}{@{}p{0.75\textwidth} p{0.25\textwidth}}
\thesisDegree & \hfill\multirow{3}{*}{\bthcsnotextlogo{3cm}} \\
\thesisMonth \ \thesisYear & \\
\end{tabular}

\begin{center}

\vspace {7.5cm}

% TITLE
{\Huge\textbf{Procedural city generation\\*[0.25cm]using Perlin noise}}

\vspace {0.5cm}

% SUBTITLE (if you have one)
% You should exclude this line, if you don't have a subtitle.
%{\Large\textbf{Thesis Subtitle}}

\vspace {3cm}

% AUTHOR NAMES
% Use the first version for single authors.
% The second version is for two authors on separate rows
%{\Large\textbf{Firstname Lastname}} % first version
{\Large\textbf{Niclas Olsson \\*[0.25cm] Elias Frank}} %second version

\end{center}

\vspace*{\fill}

\noindent%
{\small 
Faculty of Computing \\
Blekinge Institute of Technology \\
SE--371 79 Karlskrona, Sweden
}

\clearpage
} % Back to \pagestyle{plain}


% THESIS INNER PAGE (contact information to authors and supervisors go gere)
{\pagestyle{empty}
\changepage{5cm}{1cm}{-0.5cm}{-0.5cm}{}{-2cm}{}{}{}
\noindent
% Do not change the text below.
{\small
This thesis is submitted to the Faculty of Computing at Blekinge Institute
of Technology in partial fulfillment of the requirements for the degree of
\thesisDegree. The thesis is equivalent to \thesisWeeks\ 
weeks of full time studies.
}

\vspace{15cm}

\noindent
\textbf{Contact Information:} \\
Author(s): \\
Niclas Olsson \\
Elias Frank \\
E-mail:\\
niod12@student.bth.se \\
elfa14@student.bth.se

\vspace{2cm}

\noindent
University advisor: \\
Hans Tap, PhD \\
Department of Creative Technologies


\vspace*{\fill}

\noindent
\begin{tabular}{@{}p{0.5\textwidth} l c l}
Faculty of Computing             & Internet & : & www.bth.se \\
Blekinge Institute of Technology & Phone    & : & +46 455 38 50 00 \\
SE--371 79 Karlskrona, Sweden    & Fax      & : & +46 455 38 50 57 \\
\end{tabular}
\clearpage
} % Back to \pagestyle{plain}

\setcounter{page}{1}

% ABSTRACT PAGE (in \pagestyle{plain})
\abstract
\begin{changemargin}{+.5cm}{+.5cm}
\noindent
\textbf{Context}: Procedural content generation is to algorithmically generate content. This has been used in games and is an important tool to create games with large amounts of content using fewer resources. This may allow small developers to create big worlds, which makes the investigation into this area interesting.\newline
\textbf{Objectives}: The Procedural generation of cities using Perlin noise is explored. The goal is to find out if a procedurally generated city using Perlin noise is viable to use in games. \newline
\textbf{Methods}: An implementation generating cities using Perlin noise has been created and a user study along with data collection tests the cities' viability in games.\newline
\textbf{Results}: The implementation succeeds with all the technical requirements such as performance and determinism. The user study shows that the cities created are perceived as viable in games. \newline
\textbf{Conclusions}. ...

\vspace {1cm}
% You can list 3-4 keywords, maximum 2 of these from the title;
% starts 1 line below the abstract.
\noindent
\textbf{Keywords:} Procedural city generation, Perlin noise, Performance, Game content

\end{changemargin}


% TABLE OF CONTENTS PAGES (generated by LaTeX using the command(s) below)
% You should uncomment the commands you need.
\tableofcontents
%\listoffigures             % in case you have them
%\listoftables              % in case you have them
%\listofalgorithms          % in case you have them
%\include{acknowledgments}  % OPTIONAL

\cleardoublepage
\pagestyle{headings}
\pagenumbering{arabic}


% THE ACTUAL THESIS STARTS HERE
\chapter{Introduction}
\section{Introduction}
\subsection{Introduction}
Exploring a huge open world environment is a desirable feature in games. But creating a big open city such as in the Grand Theft Auto series and Batman: Arkham City involves years of work for a lot of people. Making big open cities in games is simply not feasible for most game companies. These games all have massive success with their big open worlds making a feasible generated city an attractive technique for smaller companies to be able to compete with big open worlds of their own.
\par
Content creation requires creativity and time, these are both scarce resources which is why we need technology to help with this process.
To create massive amounts of content without a big workforce algorithmically based solutions exist, Procedural Content Generation (PCG). PCG was in the past used to minimize the disk space required for games. .kkrieger is an  example of this. It has since evolved into a method to minimize workforce required for content. No man’s sky is an example of a game using PCG to minimize workforce while maximizing content.
\par
Procedural content generation is a large subject covering many different techniques. Games such as No mans sky (ref) are using PCG to generate an entire galaxy of 18 quintillion (18,446,744,073,709,551,616) (nomanssky.gamepedia.com/Planet) different planets complete with plants, animals, animations, biomes and sounds. While other games such as Borderlands (ref) uses PCG to generate over different 17,750,000 guns (borderlands.wikia.com/wiki/Weapons). Civilization IV, Minecraft and Spelunky are examples of games generating different game environments both in 2D and 3D using PCG. Minecrafts world being eight times bigger than the surface of the earth (minecraft.gamepedia.com/The\_Overworld). With enormous amounts of content like these examples we can see why handcrafting this would not be feasible nad PCG is an attractive creative method.
\par
Lots of work has been done in the field of PCG and it has been used within games for a long time, one of the earlies examples of a game using PCG is Elite(ref) from 1984. But no singular PCG solution has yet to be found, this may be attributed to the wastly different areas of use.
\par
In this thesis we are investigating the possibility of procedurally generating a city with Perlin noise that would be viable within a game. An implementation attempting to create such a city will be made and investigated through a user study along with data collection from the implementation.

\subsection{background}
Varför gör vi en pcg stad
varför just Perlin.
Finns många andra svåra tekniker.
Vi vill va snabba.

\subsection{Procedural content generation}
vad är PCG egentligen

\subsection{City generation}
Vad är en stad och varför vill vi generera den.

\subsection{Problem statement}
Vad är problemet vi försöker lösa i vår thesis.

\subsection{Objectives}
Målet med vår thesis

\subsection{Research question}
Can Perlin noise be used in a hierarchical manner to procedurally generate a city viable in games?
\\


\chapter{Related work}
\section{Theoretical Framework}
This is theoretical framework

\chapter{Theoretical framework}
\section{Noise}
Lattice noises are simple and efficient ways of generating noise\cite{TexturingModeling}. The idea behind them is to divide a coordinate system into sections spanning between every integer coordinate. These sections are called the integer lattices. By first generating pseudorandom numbers at every lattice and then interpolating between them, noise is generated.
		
\par
Value noise is a type of lattice noise. It uses a value between minus one and one at every lattice point and interpolates between them. The key difference between value noises is what type of interpolation is used. Methods such as linear and cubic interpolation have been used\cite{TexturingModeling}.
		
\par
Gradient noise differs from value noise by generating gradient vectors at each lattice point, instead of raw values. To calculate the noise value of a point, three steps are taken. First, a vector is calculated from each lattice point to the point.  Then, the dot product between this vector and the gradient vector at the corresponding lattice point is calculated. Finally, the noise value is calculated by interpolating between the dot product results.
		
\subsection{Perlin noise}
Perlin noise is a type of gradient noise. It was developed by Ken Perlin in 1983 and improved in 2002\cite{PerlinNoise}. Perlin noise works in one, two and three dimensions, but for the purpose of this explanation two-dimensional Perlin noise will be used. The algorithm expects two arguments, an \texttt{x} and a \texttt{y} coordinate. These values are normalized so as to reside between two integer lattices, creating a point. A vector from each lattice point to this point is calculated. One of the gradient vectors is chosen at random based on the normalized coordinates. The dot product between this gradient vector and the vector to the point is calculated at each lattice point. The final value is calculated by interpolating between these dot products. Perlin noise interpolates between values using a proprietary ease curve called fade. It has the form \begin{math}6t^5-15t^4+10t^3\end{math} and is graphed out in Figure \ref{fig:fade}.
		
\begin{figure}[h]
	\centering
	\includegraphics[width=0.35\linewidth]{"Images/fade"}
	\caption{Perlin's ease curve called fade.}
	\label{fig:fade}
\end{figure}
	
\section{Online vs offline}
There are two ways to use PCG algorithms, either offline or online. Online is when the content is generated while the game is being played or in a short loading screen just before the player can start playing. This allows content that is adapted to the individual player and semi-infinite content (Noor Chaker, 2016). In the game Left 4 Dead(TODO:ref) PCG is used by analyzing the players behavior and altering the experience, this could be considered a mixture of artificial intelligence and online PCG.
\par
Offline generation is when the content is being generated before the player starts playing. This could be done by the developers before releasing the game. Offline generation is useful when generating complicated content that is too slow to generate online. These are the definitions used in this thesis.
	
\section{Viability}
When developing the implementation, the following requirements were set to ensure that the result is viable in games.
	
\begin{itemize}
	\item \textbf{Validity:} The city does not have to be realistic but the city must be somewhat credible for the player to enjoy the environment. The player should not lose immersion because of broken geometry within the city. The city should also differ from a completely random city. If it is possible to distinguish a generated city from a random one, it is considered valid.
		
	\item \textbf{Determinism:} The generation should be deterministic. This means that the exact same city can be generated again with the correct seed. Nothing should be purely random. This is important so that the users have control over the output of the implementation.
		
	\item \textbf{Performance:} The city should be generated within a reasonable time frame to be viable in an online setting. A game should never have loading screens for any long period of time. This implementation have a time limit of 15 seconds as Paradox Interactive AB a game company set this as a time limit in Melins and Bengtssons collaboration about procedural generation with the company. (Constrained procedural floor
		plan generation for gameenvironments) If this limit is excedded the implementation may still be viable in a offline setting.
		
	\item \textbf{Flexibility: } The implementation should be able to generate many different viable cities with different looks.
\end{itemize} 

PCG can be either \textit{feasible} or \textit{infeasible}. For the generated content to be considered feasible it must fulfill all the constraints. (Constrained Novelty Search: A Study on Game Content Generation) Constraints are game specific but the criteria is that with all constraints fulfilled the game should be playable. For a city an example of a constraint would be that there must be a way to travel everywhere in the city, i.e. no part of the city would be blocked by houses or have no roads connected to it.
	
\section{Evaluation}
To know if the implementation does generate cities viable in games the results must be evaluated.

\chapter{Method}
\section{Method}

	\subsection{City generation method}
	Most related work focus on one of two things: procedurally generating unique building meshes(ref) or procedurally generating large cities consisting of similar looking buildings(ref). This work focuses instead on the procedurally generated placement of authored meshes in varied districts. The concept of districts breaks up the repetitive and self-similar nature of city by introducing distinct changes in appearance between them. During research three stages of generation were recognized: Districts, Blocks and Buildings. Each of these steps use the result of the previous one as starting point, in a top-down, hierarchical manner. Going through these stages is intended to generate a city that feels and looks natural.

		\subsubsection{Districts}
		The generation process starts with the generation of Districts. A District is an abstract representation of an area of the city. It controls what type of buildings it contains, the minimum and maximum height of these buildings and how densely populated the area is with these buildings. With just these few variables, you can create anything from a sparse industrial district to a dense inner city.
		
		\subsubsection{Blocks}
		The next stage in the process is the generation of Blocks. In this work, a Block is defined as an area enclosed by four roads. Looking at aerial photographs of large cities(ref) a certain pattern sometimes appears in the way roads are laid out. There is usually a few major roads stretched throughout the city and smaller, perpendicular roads connecting them together. We decided to mimic this style(TODO: Don't use we). Since a block is another abstract representation of an area, it is actually the roads that make them up. The first step of generating blocks is to generate the main roads running through the city. After the main roads have been generated, smaller roads are generated between, connecting them together. After all roads have been generated, the city has affectively been divided into a non-uniform grid. Each cell in this grid is a block. (TODO: This stage does not depend on the previous)
		
		\subsubsection{Buildings}
		The final stage is the generation of Buildings. Each building is made up of three meshes: the bottom, middle and top meshes. By dividing a building into sections, it is possible to assemble different pieces and generate new buildings. This is a powerful way to use combinatorics to generate more content. For example, authoring two buildings would result in two unique buildings. Whereas authoring two bottom, middle and top parts would result in eight unique buildings. Upon generation of a buildings, the attributes of its district is taken into consideration. What sections to combine, the minimum and maximum height as well as the density of buildings, is all dictated by the district.

	\subsection{Implementation}
	To explore the procedural generation of a city, a desktop application was created. The application provides the end user with an interface that allows for the configuration of variables along with some statistics.
	
		\subsubsection{Interface}
		Explain the interface, user input and Qt
		
		\subsubsection{Districts}
		Explain how we generate districts
		
		\subsubsection{Blocks}
		Explain how we generate blocks
		
		\subsubsection{Buildings}
		Explain how we generate buildings

	\subsection{Data collection}
	Vilken data / hur mycket / hur samlas den

\chapter{Results}
The results from the user study and the implementation are presented here. In order to measure the viability of the cities the following three requirements were set Validity, Determinism, Performance and Flexibility, definitions can be found at \ref{sec:viability}.
	
\section{User study}
A user study with 25 participants was conducted. The participants answered questions about images of procedurally generated cities from the implementation created. The user study aimed to measure the viability the cities have in a game.
	
%The user study had 25 participants. The figure \ref{fig:pie-chart-often-play} shows that most participants play games more than once a week, therefore they can be expected to have knowledge about game environments.

% how often do you play?
\par
The participants of the user study is divided into three groups: \textit{avid gamers}, \textit{occasional gamers} and \textit{non-gamers}. By dividing the participants into levels of familiarity with games, a stronger case can be made for the validity of their answers. Figure \ref{fig:pie-chart-often-play} shows that most participants play at least once a week, putting them in the group of \textit{avid gamers}. Not enough participants fell into the group of \textit{occasional} or \textit{non-gamers}. Their results have been discarded.

\begin{figure}[h]
	\centering
	\includegraphics[width=0.65\textwidth]{"Images/OftenPlay"}
	\caption{Pie chart of how often the participants play games.}
	\label{fig:pie-chart-often-play}
\end{figure}

% viable in games?
The main focus of the user study is to determine if a city generated using our application would be viable in games. The user study asks the participants which (if any) of the cities they think could be used in a game. Figure \ref{fig:pie-chart-use-in-game} shows the participants answers to this question.

\begin{figure}[h]
	\centering
	\includegraphics[width=0.7\textwidth]{"Images/UseInGame"}
	\caption{Pie chart of which cities the participants think could be used in a game.}
	\label{fig:pie-chart-use-in-game}
\end{figure}

% PCG better than random?
Under Viability (\ref{sec:viability}), two requirements are outline. A generated city must differ significantly from a city generated at random. A generated city must also avoid breaking the users immersion. In an effort to determine if our generation process fulfills these requirements, the participants were asked which city looks the most random and which city looks the most repetitive. The results are shown in Figure \ref{fig:pie-chart-random-repetitive}.

\begin{figure}[h]
	\begin{subfigure}{0.5\textwidth}
		\centering
		\includegraphics[width=0.95\linewidth]{"Images/Random"}
		\caption{Random cities chart.}
		\label{fig:pie-chart-random}
	\end{subfigure}
	\begin{subfigure}{0.5\textwidth}
		\centering
		\includegraphics[width=0.95\linewidth]{"Images/Repetitive"}
		\caption{Repetitive cities chart.}
		\label{fig:pie-chart-repetitive}
	\end{subfigure}
	\caption{Pie charts of random and repetitive cities.}
	\label{fig:pie-chart-random-repetitive}
\end{figure}

% least/most natural
Figure \ref{fig:pie-chart-natural-least-natural} shows the participants answers to the question about which city looks the least and most natural. This question was intended to clarify the impact of districts on feel of the city.

\begin{figure}[h]
	\begin{subfigure}{0.5\textwidth}
		\centering
		\includegraphics[width=0.95\linewidth]{"Images/LeastNatural"}
		\caption{Least natural cities chart.}
		\label{fig:pie-chart-least-natural}
	\end{subfigure}
	\begin{subfigure}{0.5\textwidth}
		\centering
		\includegraphics[width=0.95\linewidth]{"Images/Natural"}
		\caption{Natural cities chart.}
		\label{fig:pie-chart-natural}
	\end{subfigure}
	\caption{Pie charts of most and least natural cities.}
	\label{fig:pie-chart-natural-least-natural}
\end{figure}

% best roads
Another fundamental part of the generation process is the road network. To determine the impact that this stage has on the city, we asked the participants which of the cities had the best road network. Figure \ref{fig:pie-chart-road-network} shows the participants answers to this question.

\begin{figure}[h]
	\centering
	\includegraphics[width=0.6\textwidth]{"Images/RoadNetwork"}
	\caption{Pie chart of which cities has the best road network.}
	\label{fig:pie-chart-road-network}
\end{figure}

% ugliest/prettiest
\newpage
By asking a broad question about the look of the cities and cross referencing the result against their perceived viability in games, a correlation between look and viability can be determined. This is the intention behind asking the participants which of the cities looks the ugliest and which one looks the prettiest. Figure \ref{fig:pie-chart-ugly-pretty} shows the result of these questions.

\begin{figure}[h]
	\begin{subfigure}{0.5\textwidth}
		\centering
		\includegraphics[width=0.9\linewidth]{"Images/Ugliest"}
		\caption{Ugliest cities chart.}
		\label{fig:pie-chart-ugly}
	\end{subfigure}
	\begin{subfigure}{0.5\textwidth}
		\centering
		\includegraphics[width=0.9\linewidth]{"Images/Prettiest"}
		\caption{Prettiest cities chart.}
		\label{fig:pie-chart-pretty}
	\end{subfigure}
	\caption{Pie charts of prettiest and ugliest cities.}
	\label{fig:pie-chart-ugly-pretty}
\end{figure}

\chapter{Discussion}
\section{Data and user study analysis}
	Based on the user study it can be argued that all cities shown are perceived to be viable in games by many, however the study suggest that cities one and three are the least viable as shown in Figure \ref{fig:pie-chart-use-in-game}. City one was created with bad parameters on purpose and city three was completely random. Thus, it is not surprising that these cities seem to be perceived as least viable in games.
	
	\subsection{City 1}
		The city generated with bad parameters is the most prevalent in two parts of the user study. As seen in Figure \ref{fig:pie-chart-random} the participants in the study found it the most random city. This was unexpected since the most random city is city three, the randomly generated city. It can be seen in Figure \ref{fig:pie-chart-least-natural} that this city is seen as the least natural city by a majority of the participants. There is limited evidence, but this may show that procedurally generated cities are highly dependent upon their generation parameters. As the bad parameters generated city are perceived as both the most random and the least natural city. This city is also a contender for the ugliest city, as illustrated in Figure \ref{fig:pie-chart-ugly}.
		
	\subsection{City 2 and 4} \label{ssec:city2-3}
		Generated with similar parameters but different seeds these two cities should in theory be very similar. Two cities with similar parameters were included to see if there is any noticeable difference between similarly generated cities. These cities were hypothesized to be the most viable to use in games since they were created using the full procedural generation pipeline with parameters that were thought to be good. Figure \ref{fig:pie-chart-use-in-game} shows that these two cities are perceived as more viable than cities one and three. This was expected, however, city five seems to be even more viable. Hence, it is possible that the generation pipeline can be improved upon. As illustrated in Figure \ref{fig:pie-chart-road-network} city four seems to have much better road network than city two. Considering that they were generated with similar parameters but the results diverge to a noticeable degree it may be the case that some aspects of the cities differ more than expected. The difference between these two cities shows that cities generated with similar parameters may be very variable in some aspects even when only small changes are made to the parameters such as changing the seed.
	\subsection{City 3}
		Randomly generated, this city seems to be the least viable in games along with city one according to the user study, as shown in Figure \ref{fig:pie-chart-use-in-game}. This was expected since procedurally generated content should be more viable than randomly generated content. Where city three is the most prevalent is in the repetitive cities and ugliest cities charts, illustrated in Figure \ref{fig:pie-chart-repetitive} and \ref{fig:pie-chart-ugly}. This city is perceived as the second most random city after city one as presented in Figure \ref{fig:pie-chart-random}. The results that this city looks the most repetitive to the participants is unexpected. The least coherency and most variance in buildings should be in city three as there is no districts to control the buildings nor road variables. To analyze why city three is perceived as repetitive even though it should have the most variety is out of scope for this thesis.
	\subsection{City 5}
		Presented in Figure \ref{fig:pie-chart-use-in-game}, city five is the most viable city for games according to the user study. This is the city with randomly generated districts but otherwise use the same pipeline as cities one, two and four. Randomly generated districts affect where the different buildings may be placed along with the block size. This may result in layouts such as a house surrounded by skyscrapers. This city was seen as the most natural city by most participants, this can be seen in Figure \ref{fig:pie-chart-natural}. The participants found this city to have the second best road network after city four as shown in Figure \ref{fig:pie-chart-road-network}. This may suggest that the participants appreciate some randomness in the city but with the parameters the districts provide the house and road generation. This further supports the idea that the pipeline used in the implementation can be improved upon as suggested in \ref{ssec:city2-3}.
	\subsection{Performance}
		From section \ref{section:DataCollection} all the data collected can be seen. The generation time for 100 cells is faster than two milliseconds on average when generating the 10x10 cities. And when generating the 1000x1000 cities the generation of 100 cells is faster than 0.7 milliseconds on average. This may be a viable generation speed in an online setting. As games often aim for 60 frames per second which is 16 milliseconds, a speed of two milliseconds might therefore be viable. This generation speed is without optimizations and on a single computational thread. There are examples of impressive optimizations for city generation which shows that much optimization is possible \cite{InfiniteCities}.
		The standard deviation on the biggest city is over one second, the high standard deviation may be caused by the differences between the generated cities. One city may for example have buildings in almost all cells where others might have few buildings. This cause differences in generation time.
		
\section{Implementation analysis}
The implementation has a number of major limitations. In the interest of overview, a list of limitations that will be discussed is provided below.
	
\begin{itemize}
	\setlength\itemsep{0.01cm}
	\item Maximum of three districts.
	\item All roads are straight.
	\item One-to-one relationship between cell and building.
	\item Low quality models are used.
	\item Flat terrain.
	\item Arbitrary noise value manipulation.
\end{itemize}
	
% 3 districts
The implementation was designed from the beginning to only support three districts. This means that the implementation might not work with more districts. Limiting the number of districts to three was an effort to minimize workload while still generating interesting intersections between districts.
% straight roads
\par
Any city generated using Elicras is based on a two-dimensional grid. A cell in the grid represents a building, road or patch of grass. Depending on the model assets used when generating a city, this can look rigid and unnatural. This is the reason why all the roads are generated in straight lines. In a more believable setting, roads would be generated using curves\cite{CurvedRoads}.
% one-to-one
\par
Each cell contains at most one building. The implementation has no support for buildings that span across cells. This means that large buildings like warehouses or hangars are difficult to represent. One way to work around this limitation is to consider each cell as a large space of land. By only using small parts of the cell for small buildings and large parts of the cell for large buildings, it will appear as though there is a difference in size between cells.
% low quality models
\par
The models used in Elicras are of low quality. For most building sections simple shapes were used (see Figure \ref{fig:simple-sections}). It is likely that the overall visual fidelity of the city increases with higher quality models.

\begin{figure}[h]
	\centering
	\begin{subfigure}{0.35\textwidth}
		\centering
		\includegraphics[width=0.9\linewidth]{"Images/roof"}
		\caption{Top section of a building.}
		\label{fig:roof-mesh}
	\end{subfigure}
	\begin{subfigure}{0.35\textwidth}
		\centering
		\includegraphics[width=0.9\linewidth]{"Images/cube"}
		\caption{Mid section of a building.}
		\label{fig:cube-mesh}
	\end{subfigure}
	\caption{Building sections made up of simple shapes.}
	\label{fig:simple-sections}
\end{figure}

% flat terrain
Any city in our application is generated on a flat surface. Elicras makes no effort to consider the underlying terrain of the city. This renders our implementation ineffective in games with anything but flat terrain.
% noise manipulation
\par
Perlin noise has been used at every stage of the generation process. There are parts of the process where we have shoehorned the noise function into our implementation rather than using it to its full potential. Examples of this is when Elicras determines where to place roads or what sections to use for a building. In these areas Perlin noise is used like a pseudorandom number generator(PRNG). By using it this way, some of the desirable features of Perlin noise are neglected. In these cases, using an actual PRNG like \textit{rand}\cite{RandCRT} might produce similar results, without the implementation overhead. In other cases, the properties of Perlin noise have a visual impact on the result despite being used like a PRNG. An example of this the process of determining whether to place a building or to leave a cell vacant. By checking the noise value against a threshold, buildings are placed in a natural looking way rather than at random (see Figure \ref{fig:perlin-rand-vacant}).

\chapter{Conclusion and future work}
\section{Conclusion}
This is our conclusion

\bibliography{citations}
\bibliographystyle{plain}
\end{document}
