\section{Data and user study analysis}
	Based on the user study it can be argued that all cities shown are perceived to be viable in games by many, however the study suggest that cities one and three are the least viable as shown in Figure \ref{fig:pie-chart-use-in-game}. City one was created with bad parameters on purpose and city three was completely random. Thus, it is not surprising that these cities seem to be perceived as least viable in games.
	
	\subsection{City 1}
		The city generated with bad parameters is the most prevalent in two parts of the user study. As seen in Figure \ref{fig:pie-chart-random} the participants in the study found it the most random city. This was unexpected since the most random city is city three, the randomly generated city. It can be seen in Figure \ref{fig:pie-chart-least-natural} that this city is seen as the least natural city by a majority of the participants. There is limited evidence, but this may show that procedurally generated cities are highly dependent upon their generation parameters. As the bad parameters generated city are perceived as both the most random and the least natural city. This city is also a contender for the ugliest city, as illustrated in Figure \ref{fig:pie-chart-ugly}.
		
	\subsection{City 2 and 4} \label{ssec:city2-3}
		Generated with similar parameters but different seeds these two cities should in theory be very similar. These cities were hypothesized to be the most viable to use in games since they were created using the full procedural generation pipeline with parameters that were thought to be good. Figure \ref{fig:pie-chart-use-in-game} shows that these two cities are perceived as more viable than cities one and three. This was expected, however, city five seems to be even more viable. Hence, it is possible that the generation pipeline can be improved upon. As illustrated in Figure \ref{fig:pie-chart-road-network} city four seems to have much better road network than city two. Considering that they were generated with similar parameters but the results diverge to a noticeable degree it may be the case that some aspects of the cities differ more than expected.
	\subsection{City 3}
		Randomly generated, this city seems to be the least viable in games along with city one according to the user study, as shown in Figure \ref{fig:pie-chart-use-in-game}. This was expected since procedurally generated content should be more viable than randomly generated content. Where city three is the most prevalent is in the repetitive cities and ugliest cities charts, illustrated in Figure \ref{fig:pie-chart-repetitive} and \ref{fig:pie-chart-ugly}. This city is perceived as the second most random city after city one as presented in Figure \ref{fig:pie-chart-random}. The results that this city looks the most repetitive to the participants is unexpected. The least coherency and most variance in buildings should be in city three as there is no districts to control the buildings nor road variables. To analyze why city three is perceived as repetitive even though it should have the most variety is out of scope for this thesis.
	\subsection{City 5}
		Presented in Figure \ref{fig:pie-chart-use-in-game}, city five is the most viable city for games according to the user study. This is the city with randomly generated districts but otherwise use the same pipeline as cities one, two and four. Randomly generated districts affect where the different buildings may be placed along with the block size. This may result in layouts such as a house surrounded by skyscrapers. This city was seen as the most natural city by most participants, this can be seen in Figure \ref{fig:pie-chart-natural}. The participants found this city to have the second best road network after city four as shown in Figure \ref{fig:pie-chart-road-network}. This may suggest that the participants appreciate some randomness in the city but with the parameters the districts provide the house and road generation. This further supports the idea that the pipeline used in the implementation can be improved upon as suggested in \ref{ssec:city2-3}.
	
\section{Implementation analysis}
The implementation has a number of major limitations. In the interest of overview, a list of limitations that will be discussed is provided below.
	
\begin{itemize}
	\setlength\itemsep{0.01cm}
	\item Maximum of three districts.
	\item All roads are straight.
	\item One-to-one relationship between cell and building.
	\item Low quality models are used.
	\item Flat terrain.
	\item Arbitrary noise value manipulation.
\end{itemize}
	
% 3 districts
The implementation was designed from the beginning to only support three districts. This means that the implementation might not work with more districts. Limiting the number of districts to three was an effort to minimize workload while still generating interesting intersections between districts.
% straight roads
\par
Any city generated using Elicras is based on a two-dimensional grid. A cell in the grid represents a building, road or patch of grass. Depending on the model assets used when generating a city, this can look rigid and unnatural. This is the reason why all the roads are generated in straight lines. In a more believable setting, roads would be generated using curves\cite{CurvedRoads}.
% one-to-one
\par
Each cell contains at most one building. The implementation has no support for buildings that span across cells. This means that large buildings like warehouses or hangars are difficult to represent. One way to work around this limitation is to consider each cell as a large space of land. By only using small parts of the cell for small buildings and large parts of the cell for large buildings, it will appear as though there is a difference in size between cells.
% low quality models
\par
The models used in Elicras are of low quality. For most building sections simple shapes were used (see Figure \ref{fig:simple-sections}). It is likely that the overall visual fidelity of the city increases with higher quality models.

\begin{figure}[h]
	\centering
	\begin{subfigure}{0.35\textwidth}
		\centering
		\includegraphics[width=0.9\linewidth]{"Images/roof"}
		\caption{Top section of a building.}
		\label{fig:roof-mesh}
	\end{subfigure}
	\begin{subfigure}{0.35\textwidth}
		\centering
		\includegraphics[width=0.9\linewidth]{"Images/cube"}
		\caption{Mid section of a building.}
		\label{fig:cube-mesh}
	\end{subfigure}
	\caption{Building sections made up of simple shapes.}
	\label{fig:simple-sections}
\end{figure}

% flat terrain
Any city in our application is generated on a flat surface. Elicras makes no effort to consider the underlying terrain of the city. This renders our implementation ineffective in games with anything but flat terrain.
% noise manipulation
\par
Perlin noise has been used at every stage of the generation process. There are parts of the process where we have shoehorned the noise function into our implementation rather than using it to its full potential. Examples of this is when Elicras determines where to place roads or what sections to use for a building. In these areas Perlin noise is used like a pseudorandom number generator(PRNG). By using it this way, some of the desirable features of Perlin noise are neglected. In these cases, using an actual PRNG like \textit{rand}\cite{RandCRT} might produce similar results, without the implementation overhead. In other cases, the properties of Perlin noise have a visual impact on the result despite being used like a PRNG. An example of this the process of determining whether to place a building or to leave a cell vacant. By checking the noise value against a threshold, buildings are placed in a natural looking way rather than at random (see Figure \ref{fig:perlin-rand-vacant}).