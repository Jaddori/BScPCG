Exploring a huge open world environment is a desirable feature in games. But creating a big open city such as in the Grand Theft Auto\cite{GTA} series and Batman: Arkham City\cite{Batman} involves years of work for a lot of people. Making big open cities in games is simply not feasible for smaller game companies. These games all have great success with their big open worlds making a feasible generated city an attractive technique for smaller companies to be able to use big open worlds of their own.
\par
Content creation requires time and resources; these are both scarce, which is why game companies may want technology to help with this process. To create large amounts of content without a big workforce algorithmically based solutions exist, Procedural Content Generation (PCG). PCG was in the past used to minimize the disk space required for games. .kkrieger\cite{Kkrieger} is an  example of this. It has since evolved into a method to minimize workforce required for content. No Man's Sky\cite{NoMansSky} is an example of a game using PCG to minimize workforce while maximizing content.
\par
Procedural content generation is a large subject covering many different techniques. Games such as No Man's Sky\cite{NoMansSky} are using PCG to generate an entire universe of 18 quintillion\cite{NoMansSkyplanets} different planets complete with plants, animals, animations, biomes and sounds. While other games such as Borderlands\cite{Borderlands} uses PCG to generate over 17,750,000 different guns\cite{BorderlandsWeapons}. Civilization IV\cite{CivilizationCompany}, Minecraft\cite{Minecraft} and Spelunky\cite{Spelunky} are examples of games generating different game environments both in 2D and 3D using PCG. Minecraft's world being eight times bigger than the surface of the earth\cite{MinecraftSize}. With large amounts of content like in these examples it is understandable why handcrafting content would not be feasible and why PCG is an attractive creative method. A lot of work has been done in the field of PCG and it has been used within games for a long time. One of the earliest examples of a game using PCG is Elite\cite{Elite} from 1984. But no singular PCG solution exists, this may be attributed to the vastly different areas of use.
\par
In this thesis, the possibility of procedurally generating a city with Perlin noise, that would be viable within a game, is investigated. An implementation attempting to create such a city will be made and evaluated through a user study along with data collection from the implementation.
	
	\section{Background}
	Perlin Noise is the PCG technique chosen for every stage of the city generation. This technique was mainly chosen because of how fast it is to implement as implementation time is very important in game development.
	There are many ways to procedurally generate a city, but many of them are very complicated which often means long implementation time. From a developer's perspective there is not much use for PCG if the implementation time is equal or greater to the time it would take to handcraft every single element.
	
	\section{Procedural content generation}
	Procedural content generation in games is defined as "[...] \textit{the algorithmic creation of game content with limited or direct user input}"\cite{WhatIsPCG}. This is the definition used in this thesis. But PCG has been used for many purposes by many different people. There is no clear definition of exactly what PCG is that is universally agreed upon\cite{WhatIsPCG}.
	
	PCG is also used in areas outside the gaming industry. SpeedTree\cite{SpeedTree} is using PCG methods to generate trees that are used in movies such as Avatar\cite{SpeedTreeMovies}.
	This thesis will focus on the use of PCG in games with the previous mentioned definition.
	
	\section{City generation}
	A city has many different definitions in the real world. But a city viable in games is unlike real cities. They are not a large human settlement with governments and sanitary systems. The definition of a city explored in this thesis is "\textit{A large area occupied with different kinds of houses connected through a road system.}"
	Unlike a real city, a city viable in games do not need logic, electricity connections nor gas stations. It simply needs to be an engaging place to experience a game in. With this great simplification, generation of such a city becomes far less complex.
	To generate a city viable to use in games, three different generation stages have been recognized: Districts, blocks and buildings. These generation stages are used in the implementation.
	
	\section{Problem statement}
	How fast the content is generated and the quality of the generated content are the main problems this thesis investigates.
	To create a city purely using Perlin noise and test if it is viable in games is explored.
	
	\section{Objectives}
	Generating a city using the PCG technique Perlin noise and testing the viability in games is the goal. The time to generate a city and the cities viability within games is the corner stones explored in this thesis. The creation of an implementation is necessary as to the best knowledge of the authors there exists no such solution available.
	The following things need to be completed for the objective to be achieved.
	
	\begin{itemize}
		\item \textbf{Perlin noise:} Everything should be generated with Perlin noise.
		With only a single PCG technique to implement, the implementation should be fast to create.
		
		\item \textbf{Implementation:} An implementation that generate viable cities. This implementation will have a user interface that allows the user to rapidly generate many different cities.
		
		\item \textbf{Data collection:} The implementation must calculate and save correct data that can be analyzed. Loading times and how many times Perlin noise is used for the generation are two types of data that are relevant.
		
		\item \textbf{User study:} After the implementation is done a user study will be conducted. This user study must have atleast 20 participants answering questions from a questionnaire.
		
		\item \textbf{Conclusion: } With the data from the user study and the implementation the conclusion that Perlin noise can generate cities viable in games can be done.
	\end{itemize} 
	
	\section{Research question}
	Can Perlin noise be used to procedurally generate a city viable in games?
