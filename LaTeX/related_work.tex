\section{Related Work}
Procedural city generation has been done many times with different angles. Greuter et al\cite{PseudoInfiniteCities} created a method to procedurally generate "\textit{pseudo infinite cities}". This is done by placing the buildings along a grid. The position of each cell in the grid is hashed into a unique number. With this number the building is procedurally generated to be unique from all the other buildings. Because the position of the buildings decides how they look you can walk for a pseudo infinite time through this city, without ever encountering a building that has been seen before. To be able to render such a city, culling of geometry as well as controlling caching (loading and de-loading of assets) is important\cite{PseudoInfiniteCities}.

\par 
M\"uller proposes a method to procedurally generate a city using extended L-systems for the roads, rule based subdivision for the lots and CGA shape grammar for the geometry[reef]. This is done in stages; first the roads are generated. The roads naturally generate blocks of land. These blocks are by rule based subdivision divided into lots which fits a house generated by shape grammar as the last stage. This was done in 97.000 lines of code over 6 years of working time\cite{ProceduralModeling}\cite{ProceduralModeling6}.

\par
M\"uller et al procedurally generate a 4D city, with the fourth dimension being time. This is also done through L-systems and shape grammars. The growth of the city can be seen through time. Trying to achieve a realistic growth of a city was important here as the results are compared to real life data of city growth. \cite{4DCities} There exists even more work where cities are simulated through time\cite{AutonomousTimeVarying}.


\par
Real time generation and rendering of infinite cities using the GPU has also been done\cite{InfiniteCities}. This city is created with shape grammars. With this technique thousands of building can be rendered at 100 frames per second. Even when the camera is moving fast through the city. This is achieved with advanced culling techniques, frame-to-frame coherence, buffer management and moving important work from the CPU to the GPU.

\par
Both L-systems and CGA shape grammars for city generation have updated versions that can be used for future work\cite{InteractiveProceduralStreet}\cite{AdvancedProceduralModel}.

\par
Ken Perlin's Perlin noise was created to be used as textures for modeling, since its creation it has been updated by Perlin himself and has been widely used for other means than textures such as traffic generation\cite{PerlinNoise}\cite{TrafficGenerator}.